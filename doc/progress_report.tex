%% Define document class, variables and frontmatter
\documentclass{beamer}
\title{Progress report}
%%% Variables, functions and other settings
%% Define beamer theme
\usetheme{Goettingen}
%% Packages
%\usepackage[utf8]{inputenc}
%\usepackage[T1]{fontenc}
%% Define basic variables
\subtitle{}
\author{Neronet}
\institute[]{
  \emph{
    Toolbox for managing the training \\
    neural networks
  } \\[0.5cm]
  CSE-C2610 \\
  Software Project \\[0.2cm]
  Aalto University
}
\date{\today{}}
\subject{Software engineering}
%% Colors and visuals
% Specify link and URL colors
\hypersetup{colorlinks,linkcolor=,urlcolor=magenta}
% Replace navigation symbols with a logo
\logo{\includegraphics[height=1cm]{gfx/logo_aalto.png}}
\setbeamertemplate{navigation symbols}{\insertlogo}
%% Functions
% Define short set and clear background commands
\newcommand{\bgset}[1]{\usebackgroundtemplate{
  \includegraphics[width=\paperwidth,height=\paperheight]{#1}}}
\newcommand{\bgclear}{\usebackgroundtemplate{}}
\ifdefined \showoutlineatsection
  % Have LaTeX render an outline frame just before each new section
  %\setbeamercolor{section in toc}{fg=alerted text.fg}
  %\setbeamercolor{section in toc shaded}{fg=structure}
  \setbeamertemplate{section in toc shaded}[default][60]
  \AtBeginSection[]{
    %\bgset{gfx/neural3__bgmod.jpg}
    \begin{frame}<beamer>{Outline}
      \tableofcontents[currentsection]
    \end{frame}
    \bgset{gfx/neural1__bgmod.jpg}}
\fi
%%% Main document content
\begin{document}
%% Title page
\bgset{gfx/neural2__bgmod.jpg}
\begin{frame}
  \titlepage
\end{frame}
%% Outline page
\bgset{gfx/neural3__bgmod.jpg}
\begin{frame}{Outline}
  \tableofcontents
\end{frame}

\def\Done{\textcolor{green}{Done}}
%% Content pages
\section{Introduction} \bgset{gfx/neural1__bgmod.jpg}
\begin{frame}{Introduction}{Goals} % Joona
  Our goal is to develop a tool for computational researchers to enable easy
  \begin{itemize}
  \item specification and management of experiment queues
  \pause \item batch submission of experiment jobs to computing clusters
  \pause \item monitoring of ongoing experiments' logs and parameter values
  \pause \item access to experiment information during and after the run
  \pause \item configurable notifications on experiment state and progress
  \pause \item configurable criteria for experiment autotermination
  \pause \item logging of experiment history
  \end{itemize}
\end{frame}
\begin{frame}{Introduction}{What} % Samuel
  In essence the product is a Python-based tool that enables computational researchers to conduct their research more effectively.
  \begin{itemize}
  \item \pause It utilizes SSH and TCP sockets to distribute the computational workload into computer clusters. It supports the Slurm job and resource manager but can function without it as well.
  \item \pause It is framework agnostic in that it permits the use of a very wide variety of tools to actually conduct the computing.
  \end{itemize}
\end{frame}
\section{Results} % Samuel & Joona
\begin{frame}{Results}{Sprint 0} \bgset{gfx/neural3__bgmod.jpg}
  Goal: Team building and preparing for sprint 1 \pause \Done{}
  
  \pause Product Backlog Items: \emph{None}

  \pause Sprint 0 took a lot of effort from us since the project topic was very challenging to dive into. Also none of us had done this course before. Interviews with Jelena \& Simo helped us to understand the project.
  
  \pause We were proud of our efforts in the sprint.
\end{frame}
\begin{frame}{Results}{Sprint 1}
  Goal: Develop a prototype that offers the most basic functionality via a CLI \pause \Done
  
  \pause Product Backlog Items: 
  \begin{itemize}
  \pause \item US1: As a user, I want a basic user guide that would cover the installation of Neronet and its use via CLI. \pause \Done{}
  \pause \item US2: As a user, I want to specify clusters by address and type to specify my computing resources. \pause \Done{}
  \pause \item US3: As a user, I want to specify experiments by name, files and parameters and edit and delete them. \pause \Done{}
  \pause \item US4: As a user, I want to submit experiments to unmanaged nodes. \pause \Done
  \pause \item US5: As a user, I want an experiment status report so that I can review experiment status details. \pause \Done{}
  \end{itemize}

  \pause Just a prototype, a lot of work to do before user testing.
\end{frame}
\section{Demo}
\begin{frame}{Demo} % Iiro & Teemu
  Demo script:
  \begin{enumerate}
  \item Neronet Installation, preferences and initial setup of clusters
  \item Specification of clusters via CLI
  \item Specification of an experiment
  \item Submission of the specified experiment to an unmanaged node
  \item Retrieval of experiment status report
  \end{enumerate}
\end{frame}
\section{Quality} \bgset{gfx/neural1__bgmod.jpg}
\begin{frame}{Quality} % Matias & Tuomo
  Definition of done:
  \begin{itemize}
  \item We defined \Done{} in three levels: BI, sprint and project
  \item BI level: unit tests done where applicable,
    functional test coverage 80\%, conformity (PEP-8), commented,
    documented, peer reviewed
  \item Sprint level: BI:s are \Done{}, increment is tested and reviewed,
    sprint goal is achieved
  \item Updates to DoD:
    \begin{itemize}
    \item We replaced \emph{unit test coverage 90\%} with \emph{unit tests
      are written where applicable} -- the old metric was not useful for all
      BIs
    \end{itemize}
  \item Otherwise, we have followed our DoD as planned.
  \end{itemize}
\end{frame}
\begin{frame}{Quality} % Matias & Tuomo
\begin{table} \small
  \begin{tabular}{l|rrrrrrrrrrr}
    US &   UTC &   FTC &   Com &   Doc &   Rev  \\ \hline
     1 &     3 &     3 &     3 &     3 &     3  \\
     2 &     2 &     2 &     3 &     2 &     3  \\
     3 &     3 &     3 &     3 &     3 &     3  \\
     4 &     3 &     3 &     3 &     3 &     1  \\
     5 &     1 &     3 &     2 &     1 &     1  \\ \hline
       &     2 &     3 &     3 &     2 &     2
  \end{tabular}
  \end{table}
  Qualitatively we achieved our standards only partially:
  \begin{itemize}
  \item Unit and functional test coverage -- good
  \item Quality of comments and documentation -- good
  \item Peer review -- ok (done rather quickly)
  \end{itemize}
\end{frame}
\begin{frame}{Quality} % Matias & Tuomo
  Used QA practices and tools:
  \begin{itemize}
  \item Commenting \& documentation -- forces to rethink from another
    perspective, facilitates peer review
  \item Python standard unittest framework -- test automation
  \item Peer review -- quality assurance
  \end{itemize}
\end{frame}
\begin{frame}{Quality} % Matias & Tuomo
  Relevant quality attributes:
  \begin{itemize}
  \item Usability -- We developed a basic user guide in the first sprint
    which will help even newbies understand our software $\rightarrow$ The
    usability of our software should be good
  \item Reliability -- Unfortunately, we didn't have as much time to test our
    software in the first sprint as we'd hoped. We will make up for this by
    using more of our second sprint for testing and less for making new
    features
  \item Extendability -- At the moment our software's extendability is ok,
    difficult to say anything about the final product
  \item Performance -- At the moment, our software's performance is ok
  \end{itemize}
\end{frame}
\section{Effort}
\begin{frame}{Effort} % Matias & Tuomo
\begin{table} \small
  \begin{tabular}{l|rrrrrrrrrrr}
    S &      Sa &      Te &      Tu &      Jo &      Ii &      Ma  \\ \hline
    0 &  140/50 &   36/35 &   45/35 &   40/35 &   36/35 &   43/35  \\
    1 &   46/30 &   28/33 &   33/33 &   38/33 &   25/33 &   33/33  \\
    2 &    0/30 &    0/33 &    0/33 &    0/33 &    0/33 &    0/33  \\
    3 &    0/15 &    0/33 &    0/33 &    0/33 &    0/33 &    0/33  \\
    4 &    0/15 &    0/33 &    0/33 &    0/33 &    0/33 &    0/33  \\
    5 &    0/15 &    0/33 &    0/33 &    0/33 &    0/33 &    0/33  \\
    6 &    0/20 &    0/25 &    0/25 &    0/25 &    0/25 &    0/25  \\ \hline
      & 186/175 &  64/225 &  78/225 &  78/225 &  61/225 &  76/225  \\
  \end{tabular}
  \end{table}
\end{frame}
\section{Retros} \bgset{gfx/neural3__bgmod.jpg}
\begin{frame}{Retros: Sprint 0} % Joona & Samuel
  Sprint planning:
  \begin{itemize}
  \item backlog items must be clear and simple -teemu
  \item backlog items have been unclear, but the user guide probably helps
  \item it would have been better if the PO had created the stories from scratch -matias, tuomo
  \item the PO should give input when developing the user guide
  \item we should make sure we reserve enough time for the actual story selection on Monday -matias
  \end{itemize}
\end{frame}
\begin{frame}{Retros: Sprint 0} % Joona & Samuel
  Daily scrums:
  \begin{itemize}
  \item we have mostly been doing teamwork, so there has been little new info in the scrums -Matias -Joona -Teemu
  \item they have been overly long and they have extended due to inexperience.
  \item people are late.
  \end{itemize}
\end{frame}
\begin{frame}{Retros: Sprint 0} % Joona & Samuel
  Teamwork sessions:
  \begin{itemize}
  \item sessions are too long and sometimes people get hungry.
  \item generally someone has to leave early or comes late
  \end{itemize}
\end{frame}
\begin{frame}{Retros: Sprint 0} % Joona & Samuel
  Tools:
  \begin{itemize}
  \item flowdock is good x6
  \item for remote work we have been using google hangout and skype. Skype has proven to be the most stable.
  \item for faster communication we are using whatsapp.
  \item agilefant has a steep learning curve. -Iiro
  \item people tend to forget to log their time at agilefant.
  \item hope to use more github during sprints
  \item floobits ain't very good. Doesn't seem to work in its intended purpose.
  \item Top 3 tools: 1) GitHub 2) Flowdock 3) Agilefant
  \item Worst 3 tools: 1) Floobits 2) Six tactics 3) Agilefant
  \end{itemize}
\end{frame}
\begin{frame}{Retros: Sprint 0} % Joona & Samuel
  How teamwork could be improved:
  \begin{itemize}
  \item People should be more on time.
  \item hard to think on improvements on sprint 0
  \end{itemize}
\end{frame}
\end{document}