%% Define document class, variables and frontmatter
\documentclass[paper=a4,parskip,fontsize=12pt]{scrreprt}
\def\firstname{Samuel}
\def\lastname{Marisa}
\def\studentid{79770K}
%% Package Imports
\usepackage{lmodern} % specify Latin Modern as the font
\usepackage[T1]{fontenc}
\usepackage{textcomp} % fixes problems with some non ASCII chars like € \usepackage{pxfonts}
\usepackage[utf8]{inputenc} % enhanced UTF8 support (check also: utf8x)
\usepackage[english]{babel} % provides language specific hyphenation
\usepackage[headtopline=2pt,headsepline=1pt,footsepline=1pt]{scrlayer-scrpage} % provides footer/header and con­fig­urable page styles
\usepackage[protrusion=true,expansion=true,tracking=true,factor=1000,shrink=50]{microtype} % enhance the general appearance of the text
\usepackage[pdftex]{graphicx} % extends the graphics package (f. ex. key-values to \in­clude­graph­ics)
\usepackage[svgnames,x11names,hyperref]{xcolor} % provides enhanced color functionality (see http://www.latextemplates.com/svgnames-colors)
\usepackage{hyperref} % for enhanced referencing and \hypersetup for PDF metadata
\usepackage{enumerate} % enhanced list enumeration styling
\usepackage{lastpage} % for \pageref{LastPage}
\usepackage{listings} % provides source code listing cabability
\usepackage{enumitem} % enhanced customization for itemize, enumerate and description environments (\setlist)
%% Variables
\title{Learning diary}
\author{\student}
\date{\today{}}
\makeatletter
%% Headers & Footers
\setkomafont{pagehead}{\normalfont\small} % normalsize, small, footnotesize, scriptsize, tiny
\setkomafont{pagefoot}{\normalfont\small}
\chead{}
\ihead{\@title}
\ohead{\@author / Team 11 / \@date}
\cfoot{\thepage{} (\pageref{LastPage})}
\ifoot{}
\ofoot{}
%% Hyperref Settings
\hypersetup{
  unicode = true, % non-Latin characters in Acrobat?s bookmarks
  pdfstartview = {FitH}, % fits the width of the page to the window
  pdftitle = {\@title},
  pdfauthor = {\@author},
  pdfsubject = {\@title},
  pdfcreator = {LaTeX},
  pdfproducer = {\@author},
  pdfkeywords = {diary} {software development} {scrum},
  pdfnewwindow = true, % links in new window
  colorlinks = true, % false: boxed links; true: colored links
  linkcolor = MidnightBlue, % color of internal links
  citecolor = MidnightBlue, % color of links to bibliography
  filecolor = MidnightBlue, % color of file links
  urlcolor = MidnightBlue % color of external links
}
%% Other Settings
% Fonts, text flow and characters
% linespread{1.25} \KOMAoptions{DIV=last} % to alter line spacing
\setkomafont{disposition}{\bfseries}
\DeclareUnicodeCharacter{20AC}{\texteuro{}}
\DeclareUnicodeCharacter{00A0}{\texteuro{}}
\DeclareUnicodeCharacter{B0}{\textdegree}
\setlist{nosep} % eliminate vertical spacing between itemize elements
%% The Document
\pagestyle{scrheadings}
\begin{document}
%% Content pages

\chapter{Scrum game}

This entry was written on October 8, 2015 a week after my participation in
the scrum game with my team. Here I'll answer the questions put forward by
the organizers:

\emph{What were the most important things that you learned? Discuss
critically what you learned during the game (and mention if you felt you
did not learn something)}

\begin{itemize}
\item I learned through practice that perhaps the most important function
  of sprint planning for the development team is to discuss and unify
  understanding regarding the work items that are identified in the sprint
  backlog and prioritized by the PO.
\item Straightforward, effective communication is key to success in almost
  every part of Scrum.
\item Preparing to work without permission to discuss emphasized the
  importance of planning and careful allocation of responsibilities. This
  is relatively easy in projects where the future work content is easy to
  understand and thus plan such as in the case of building something
  everybody has a pretty decent understanding of with Legos. Repeated
  sharing of understanding and redirection of plans at short intervals is
  likely much more important when the work is less understood.
\item I got to experience how the scrum master can behave in practice and how
  the interaction between him/her, the PO and the rest of the team can work
  in a pretty generic setting.
\item I relearned how easy it is to misunderstand requirements and thus
  waste time and how tricky asking good questions about requirements can be.
  Prototyping is a very effective way to alleviate this problem as it often
  works as a great tool for communicating requirements, priorities and ideas
  for potential solutions.
\end{itemize}

\emph{Do you think that the game will help your project succeed? How?}

I am certain that the game experience will help our project succeed because:

\begin{itemize}
\item Of the important learning I already explained above.
\item For me personally, getting a practical example of how Scrum can work
  in practice is very important and facilitates my job learning to be a good
  scrum master
\item For my team, it is very important that they got an idea of how Scrum
  works and why it is important. It will help me when trying to motivate
  them to respect its principles, procedures, events and artifacts — the
  "overhead" it brings.
\item It also worked as a quick and effective way to build some sense of team
  unity.
\end{itemize}

\emph{You may also comment on the game, what was good/what could be
improved, any feedback on game is very welcome!}

\begin{itemize}
\item The game was good, but it would have been interesting also if there
  would have been the added complexity that 1 person at each iteration would
  switch the team. Thus requiring the exercise of additional communication
  for the new member but also to give people the chance to see how different
  teams might work differently. Of course that might require 7 sprints for
  teams of 7 members for everyone to see another team which is why it might
  not be always feasible.
\end{itemize}

\chapter{Project review I}

This entry was written on December 5, 2015 just after our first project
review. First I'll describe three of the educational observations I have
made during the first third of the project. Second, I'll summarize my
contributions to the project so far.

First, Scrum seems to be a very reasonable software development methodology.
It is lightweight yet it provides a suprisingly robust basic structure and
system to build on. While giving "top-down" some structure it manages to
acknowledge the human aspects of teams and team members and to not restrict
free thinking by too many rules and division of roles, responsibilities and
processes. It seems to really do what it claims to do: emphasize the
importance of and facilitate performance in transparency, inspection, and
adaptation. Moreover, it leaves room for the varying personalities,
skillsets and desires of team members.

Second, in the beginning I had to study and prepare a lot for each scrum
event, but currently I feel very little stress about them. The two events
that are still challenging and require more inspection are daily scrums and
retrospectives, perhaps because I saw them as the least challenging early on.
Eliciting meaningful but brief output from myself and our developers in
daily meetings has not been easy, though we have been doing visible progress
in the application of the practice.

So far we've been summarizing everybody's daily scrum output into our meeting
notes file. This formality and the requirement that we don't finish until
we have logged some meaningful output from each member was necessary in my
opinion in order to make sure everyone gets used to really following the
practice.

In the future we might stop doing that and make the scrums shorter, expecting
that developers are more able to bring up noteworthy stuff without any
elicitation effort. The same holds true for our retrospectives. In the future
we will likely make the retrospective sessions slightly more compact
expecting that team members have already thought about relevant stuff prior
to the session and are more used to giving out feedback and sharing their
thoughts. I plan to also add some more structure, perhaps some sort of a
discussion frame (or game), so that most important areas are dealt with
effectively.

My third observation is about a problem with Scrum that we have faced. In my
opinion the successful application of Scrum seems to require that team
members are not only devoted, but both skilled and experienced.

According to the course material the development team should be self
organizing and independent. However, in my opinion a group of inexperienced
students such as in our team cannot be expected to very effectively organize
themselves quickly to immediately start yielding top results. I feel that
there should preferably be at least two energetic individuals per development
team to boost the pace of the group and at least two people with strong
knowhow and experience in relevant areas in order for the team to be able to
smoothly self organize itself, study the problem, design an architecture of
the solution, and quickly start generating quality software.

Our team had limited software development experience outside the basic Aalto
programming courses. Similarly, I found that there was only a limited amount
of the stereotypic American MBA graduate spirit among the personalities of
our team. I felt that the potential of our team members could be effectively
drawn forward only by a strong leader, almost a top-down director at times,
and that the team members actually wanted that kind of clear leadership.

Also, not all members had the intrinsic thirst for achieving great success
in general. Several members seemed to be fine with just about passing the
course. We managed to heal this divergence by organizing a few sessions where
we discussed the benefits of excelling at the project, what it would take,
and how realistic it would be. I shared some experiences I had had and also
presented a set of slides based upon scientific research on successful
teamwork presented to me on another Aalto course.

Anyhow, subsequently I have felt that I have been forced to take up roles of
not only the scrum master but also the communicator (all stakeholders), the
director (planning, scheduling, top-down direction), the team leader
(motivator, psychological support), the architect (overall design) and the
lead developer and doer (plans, artifacts, harder programming challenges).

This uneven experience and power structure unsurprisingly produced some
conflict and inefficiency in that some team members often repeatedly
questioned my recommendations and argumentation. Consequently, a lot of time
had to be spent also on communicating why and how I saw that certain
non-technical and technical things should be done in certain ways.

At the end of sprint 0 I proposed to designate the role of sprint team leader
to partly alleviate this problem. Mainly to share the prementioned roles of
the communicator, the director and the team leader. I proposed that Joona
would take the role as I saw him as the most ambitious and driven and thus
the safest candidate. I thought that he would also likely give a good example
to the others, as he did.

I need not mention that I and we had to put a lot of effort during sprint 0
to study and learn about the requirements, problem domain and potential
solutions related to the PO's vision. A basic web software project would
have been a lot easier since in addition to me one other student in our team
had done the Aalto web software development course and some had related
experience while I was practically the only one with Linux sysadmin
experience. Architecturally web software projects are also somewhat simpler
to start with because they almost always are built on a standard MVC
framework for which there are plenty of clear tutorials and guides like those
for Ruby on Rails and Django. In our case there was no one with solid
software development experience that would think for us how to approach
the problem. Instead we had to design everything from the ground up and find
relevant third party experts to consult us. Summa summarum, I considered and
still cońsider the project together with its full scale of problem areas from
technical to social both really interesting and really challenging.

In the end, in my opinion, we still managed to proceed pretty effectively
overall. Joona did an awesome job during sprint 1 as sprint team leader and
I think the team is now a lot stronger after overcoming and discussing the
many challenges this project has confronted us with so far. I might have
been somewhat ineffective with my time use, but I feel that I mostly only
tried my best to do what I thought was necessary as fast as I could. I had
lots of other business to attend to during this fall so spending my time was
no problem. However, I have to admit that I enjoyed the project, and even
writing this diary, which partly explains my effort.

The following depicts my main contributions since the start of the project:

\begin{enumerate}
\item Acted as the team's scrum master and leader -- motivator, facilitator,
  driver -- lead doer.
\item Teached and helped the team to do larger software projects as a team
  (building, studying and setting up services and tools like GitHub, git-flow,
  Floobits, Flowdock, Agilefant, Python virtual environments, Sphinx,
  Python unit test framework, Markdown, ReStructuredText, LaTeX).
\item Helped the team understand the project, design the solution and
  develop it.
\item Prepared initial versions of the scrum artifacts.
\item Programmed the initial prototype for sprint 0 review and solved
  critical programming problems during sprint 1.
\item Did several other things I saw were necessary in order to make sure
  we excel as a team.
\end{enumerate}
\end{document}