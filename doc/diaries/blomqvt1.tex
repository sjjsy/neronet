%% Define document class, variables and frontmatter
\documentclass[paper=a4,parskip,fontsize=12pt]{scrreprt}
\def\firstname{Teemu}
\def\lastname{Blomqvist}
\def\studentid{222804}
%% Package Imports
\usepackage{lmodern} % specify Latin Modern as the font
\usepackage[T1]{fontenc}
\usepackage{textcomp} % fixes problems with some non ASCII chars like € \usepackage{pxfonts}
\usepackage[utf8]{inputenc} % enhanced UTF8 support (check also: utf8x)
\usepackage[english]{babel} % provides language specific hyphenation
\usepackage[headtopline=2pt,headsepline=1pt,footsepline=1pt]{scrlayer-scrpage} % provides footer/header and con­fig­urable page styles
\usepackage[protrusion=true,expansion=true,tracking=true,factor=1000,shrink=50]{microtype} % enhance the general appearance of the text
\usepackage[pdftex]{graphicx} % extends the graphics package (f. ex. key-values to \in­clude­graph­ics)
\usepackage[svgnames,x11names,hyperref]{xcolor} % provides enhanced color functionality (see http://www.latextemplates.com/svgnames-colors)
\usepackage{hyperref} % for enhanced referencing and \hypersetup for PDF metadata
\usepackage{enumerate} % enhanced list enumeration styling
\usepackage{lastpage} % for \pageref{LastPage}
\usepackage{listings} % provides source code listing cabability
\usepackage{enumitem} % enhanced customization for itemize, enumerate and description environments (\setlist)
%% Variables
\title{Learning diary}
\author{\student}
\date{\today{}}
\makeatletter
%% Headers & Footers
\setkomafont{pagehead}{\normalfont\small} % normalsize, small, footnotesize, scriptsize, tiny
\setkomafont{pagefoot}{\normalfont\small}
\chead{}
\ihead{\@title}
\ohead{\@author / Team 11 / \@date}
\cfoot{\thepage{} (\pageref{LastPage})}
\ifoot{}
\ofoot{}
%% Hyperref Settings
\hypersetup{
  unicode = true, % non-Latin characters in Acrobat?s bookmarks
  pdfstartview = {FitH}, % fits the width of the page to the window
  pdftitle = {\@title},
  pdfauthor = {\@author},
  pdfsubject = {\@title},
  pdfcreator = {LaTeX},
  pdfproducer = {\@author},
  pdfkeywords = {diary} {software development} {scrum},
  pdfnewwindow = true, % links in new window
  colorlinks = true, % false: boxed links; true: colored links
  linkcolor = MidnightBlue, % color of internal links
  citecolor = MidnightBlue, % color of links to bibliography
  filecolor = MidnightBlue, % color of file links
  urlcolor = MidnightBlue % color of external links
}
%% Other Settings
% Fonts, text flow and characters
% linespread{1.25} \KOMAoptions{DIV=last} % to alter line spacing
\setkomafont{disposition}{\bfseries}
\DeclareUnicodeCharacter{20AC}{\texteuro{}}
\DeclareUnicodeCharacter{00A0}{\texteuro{}}
\DeclareUnicodeCharacter{B0}{\textdegree}
\setlist{nosep} % eliminate vertical spacing between itemize elements
%% The Document
\pagestyle{scrheadings}
\begin{document}
%% Content pages

\chapter{Project review I}

In sprint 1 our team decided to work on our user stories in pairs. With 5
developers that amounted to two pairs and one person doing work solo. I thought
this was an okay idea in theory, but we hadn't really discussed how we would
implement pair work in practice. A problem we had with my pair was that we tried
to focus on one user story at a time, but the user stories we implemented were
too small for us to split the work properly, and often I had no part to work on.
As a result I spent a lot of time not coding myself but simply trying to help my
partner with whatever problems arised. In hindsight this was still somewhat
useful as I was fully aware of how my partner implemented each of the functions,
and we solved problems faster. On the other hand, I think we could have used our
time more efficiently. In an optimal scenario we would both have something to
work on at all times, and we could ask each other for help on problems and then
from time to time would sync up on what the other has done (also possibly asking
the other for clarification on his own code) so that both are fully aware of how
all parts work. If we still keep on working in pairs this will be something to
remember.

Our team didn't really utilize the advantages of our work practices. We do all
our work in the same space, presumably to allow easy communication. When it came
time to merge the parts of our work that needed to work together, both pairs had
no idea on how the other pair's part worked and thus merging them properly
became difficult. In hindsight we had a lot of time to work out the necessary
interfaces between our two parts, but both pairs were simply focused on getting
their part of the work done. This is something we should obviously improve.

In sprint 0 we spent most of the time trying to figure out what would be the
best way to implement our system, and while we did end up using a lot of man
hours on that, it did help a lot in the development, as everyone now knows on a
high level of abstraction how the system is supposed to work. I don't think we
could have really started developing a usable system without that knowledge.

My contributions: I did 3 user stories with my pair: - Being able to specify
experiments - Being able to specify clusters - Getting a status report of all
the defined clusters and experiments

\end{document}
