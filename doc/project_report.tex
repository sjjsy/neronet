%% Define document class, variables and frontmatter
\documentclass{beamer}
\usepackage{tikz}
\title{Project report}
\def\showoutlineatsection{true}
%%% Variables, functions and other settings
%% Define beamer theme
\usetheme{Goettingen}
%% Packages
%\usepackage[utf8]{inputenc}
%\usepackage[T1]{fontenc}
%% Define basic variables
\subtitle{}
\author{Neronet}
\institute[]{
  \emph{
    Toolbox for managing the training \\
    neural networks
  } \\[0.5cm]
  CSE-C2610 \\
  Software Project \\[0.2cm]
  Aalto University
}
\date{\today{}}
\subject{Software engineering}
%% Colors and visuals
% Specify link and URL colors
\hypersetup{colorlinks,linkcolor=,urlcolor=magenta}
% Replace navigation symbols with a logo
\logo{\includegraphics[height=1cm]{gfx/logo_aalto.png}}
\setbeamertemplate{navigation symbols}{\insertlogo}
%% Functions
% Define short set and clear background commands
\newcommand{\bgset}[1]{\usebackgroundtemplate{
  \includegraphics[width=\paperwidth,height=\paperheight]{#1}}}
\newcommand{\bgclear}{\usebackgroundtemplate{}}
\ifdefined \showoutlineatsection
  % Have LaTeX render an outline frame just before each new section
  %\setbeamercolor{section in toc}{fg=alerted text.fg}
  %\setbeamercolor{section in toc shaded}{fg=structure}
  \setbeamertemplate{section in toc shaded}[default][60]
  \AtBeginSection[]{
    %\bgset{gfx/neural3__bgmod.jpg}
    \begin{frame}<beamer>{Outline}
      \tableofcontents[currentsection]
    \end{frame}
    \bgset{gfx/neural1__bgmod.jpg}}
\fi
%%% Main document content
\begin{document}
%% Title page
\bgset{gfx/neural2__bgmod.jpg}
\begin{frame}
  \titlepage
\end{frame}
%% Outline page
\bgset{gfx/neural3__bgmod.jpg}
\begin{frame}{Outline}
  \tableofcontents
\end{frame}

\def\Done{\textcolor{green}{Done}}
\def\Undone{\textcolor{red}{Undone}}
%% Structure
% ---- INTRO
% #) Pyry's statement as an interest booster: best group so far, professional results
% #) Team -- team photo, description
% #) Pyry -- photo of Pyry, role, background
% ---- PROJECT
% #) Pyry's problem; our challenge
% #) Vision
% #) Technical challenges, ASRs
% #) Solution -- architecture
% #) Outcome summary: what Neronet can do now?
% #) Demo
% ---- PROCESS
% #) Process overview; schedule; used tools
% #) Definition of done
% #) Testing and QA
% #) Overview of sprints: goals
% #) Stories done by sprint and sprint goals; challenges and found solutions
% #) Peer team feedback
% #) User feedback
% #) Quality attributes
% #) Retrospective summaries
% #) Overall project summary
% #) Overall project challenges: Pyry, topic, experience
% #) Summary of technical and sprint challenges and solutions
% #) Future as an OS project
% #) Thanks to our team, Eero, Pyry, Simo, Jelena, course staff
% Notes:
% - Lots of content: We have to train a lot in order to fit it all into 35 minutes.
% - Key things to mention must be printed on paper
%% Content pages


\begin{frame}{Forecitation}
  "I have supervised several bachelors' theses and other student projects.

  \vspace*{1cm}
  You and the product surpassed my expectations.
  I did not expect such professional results."

  \vspace*{1cm}
  Pyry Takala, PO
\end{frame}

\section{People} \bgset{gfx/neural2__bgmod.jpg}

\begin{frame}{Team}
  \begin{figure}
    \centering\includegraphics[width=1\columnwidth]{gfx/team.jpg}
  \end{figure}
  At the start of the course, we were all both
  \begin{itemize}
  \item \alert{terrified}, and 
  \item \alert{inspired}
  \end{itemize}
  by our project topic!
\end{frame}
\begin{frame}{Product owner}
  \begin{figure}
    \centering\includegraphics[width=0.4\columnwidth]{gfx/pyry.jpg}
  \end{figure}
  Pyry Takala
  \begin{itemize}
  \item Machine learning researcher at Aalto
  \item Past: Amazon, McKinsey \& Company, Goldman Sachs, August...
  \end{itemize}
\end{frame}
\begin{frame}{Agile coach}
  \begin{figure}
    \centering\includegraphics[width=0.4\columnwidth]{gfx/eero.jpg}
  \end{figure}
  Eero Laukkanen
  \begin{itemize}
  \item Continuous delivery researcher at Aalto
  \item Past: Prosys PMS Ltd, WiRCA-miehet, Aalto...
  \end{itemize}
\end{frame}

\section{Project} \bgset{gfx/neural1__bgmod.jpg}

\begin{frame}{Pyry's problem}
  Challenges in training neural networks:
  \begin{itemize}
  \item Long duration of individual experiments
  \item Difficult to manage of experiment variations, queues and computing nodes
  \item Difficult to monitor and analyse experiment progress
  \item Difficult to log experiment histories
  \end{itemize}
\end{frame}
\begin{frame}{Ideas}
  Could a toolbox be developed that would
  \begin{itemize}
  \item monitor experiment logs and detect problems
  \item notify of and/or autoterminate poorly developing experiments
  \item facilitate experiment variation
  \item faciliate experiment and queue management
  \item enable easy access to experiment data
  \item potentially also visualize parameter evolution
  \end{itemize}
\end{frame}
%\begin{frame}{Vision}
%  A tool that would
%  \begin{itemize}
%  \item facilitate experiment management
%  \item provide information on ongoing experiments
%  \item be able to autoterminate experiments
%  \item archive all results
%  \item possibly even facilitate analysis
%  \end{itemize}
%\end{frame}
\begin{frame}{Ideas}
  Such a toolbox should also be
  \begin{itemize}
  \item lightweight
  \item easily extensible
  \item open source
  \item framework agnostic
  \item convenient in practice
  \end{itemize}
\end{frame}
\begin{frame}{Journey map}
  \begin{figure}
  \tikzset{
    jmi/.append style={align=center,font=\tiny,shape=circle,draw,fill=gray!30,
    minimum size=1.4cm}, jmc/.style={font=\normalsize},
    man/.style={fill=blue!30}, mum/.style={draw=red,thick},
    kid/.style={fill=green!30}, nm/.style={color=blue},
    nj/.style={color=green}, jm/.style={color=red}
  }
  \resizebox{0.8\columnwidth}{!}{
  \begin{tikzpicture}
    \node (00)            [jmi,jmc]          {Research \\ journey};
    \node (01) at (180:4) [jmi]              {Innovate \\ ideas};
    \node (02) at (150:4) [jmi]              {Review \\ literature};
    \node (03) at (120:4) [jmi]              {Write \\ code};
    \node (04) at ( 90:4) [jmi,man]          {Embed \\ analytics};
    \node (05) at ( 60:4) [jmi,man]          {Design \\ tests};
    \node (06) at ( 30:4) [jmi,man,mum]      {Launch \\ tests};
    \node (07) at (  0:4) [jmi,kid,mum]      {Check \\ progress};
    \node (08) at (330:4) [jmi,kid,mum]      {Abort \\ failures};
    \node (09) at (300:4) [jmi,kid,mum]      {Collect \\ results};
    \node (10) at (270:4) [jmi,man]          {Analyse \\ results};
    \node (12) at (225:4) [jmi]              {Publish/drop \\ results};
    \foreach \a/\b in {00/01,01/02,02/03,03/04,04/05,05/06,06/07,07/08,08/09,09/10,10/01,10/12,12/00}
      \draw [->] (\a) edge (\b);
    \node (nm) at ( 30:2) [nm]               {\texttt{neroman}};
    \node (nm) at (  0:2) [jm]               {\texttt{neromum}};
    \node (nj) at (330:2) [nj]               {\texttt{nerokid}};
  \end{tikzpicture}
  }
  \end{figure}
\end{frame}
\begin{frame}{Technical challenges \& ASRs} % Joona
  \begin{itemize}
  \item deeply technical and an advanced domain
  \item remote computing nodes and cluster management
  \item good usability? according to whom?
  \item framework agnostic?
  \end{itemize}
\end{frame}

\section{Results}

\begin{frame}{Result}
  A three component tool involving Python, SSH and sockets
  that ... \alert{meets the vision}!

  \begin{itemize}
  \item All core functionality \Done{}.
  \item Most features accessible both via CLI and GUI.
  \item PO satisfied.
  \item We are proud!
  \end{itemize}
\end{frame}
\begin{frame}{Solution}
  \begin{figure}
  \tikzset{
    ndom/.style={align=left}
    ncmp/.style={align=center,font=\small,shape=circle,draw,minimum size=3cm},
    ncmp/.style={align=center,font=\small,shape=circle,draw,minimum size=3cm},
    nmum/.style={align=center,font=\small,shape=circle,draw,minimum size=2cm},
    njob/.style={align=center,font=\tiny,shape=circle,draw,fill=green!30},
  }
  \resizebox{0.8\columnwidth}{!}{
    \begin{tikzpicture}
      \draw[dashed] ( 0, 6  ) rectangle (11,11);
      \draw[dashed] ( 0, 0  ) rectangle (11, 5);
      \node (ln) at ( 2,10  ) [ndom]                 {Local network};
      \node (ln) at ( 2, 4  ) [ndom]                 {Cluster network};
      \node (nm) at ( 6, 9  ) [ncmp,fill=blue!30]    {\texttt{neroman}};
      \node (mm) at ( 6, 5.5) [nmum,fill=red!30]     {\texttt{neromum}};
      \node (j1) at ( 3, 2  ) [njob]                 {\texttt{nerokid}};
      \node (j2) at ( 5, 1  ) [njob]                 {\texttt{nerokid}};
      \node (j4) at ( 7, 1  ) [njob]                 {\texttt{nerokid}};
      \node (j3) at ( 9, 2  ) [njob]                 {\texttt{nerokid}};
      \foreach \a/\b in {nm/mm,mm/j1,mm/j2,mm/j3,mm/j4}
        \draw [<->] (\a) edge (\b);
    \end{tikzpicture}
  }
  \end{figure}
\end{frame}
\begin{frame}{Structural architecture}
  \begin{figure}
    \centering
    \includegraphics[width=0.8\columnwidth]{gfx/structural_architecture.png}
  \end{figure}
\end{frame}
\begin{frame}{Features}
  Core features:
  \begin{itemize}
  \item extensible experiment specification system
  \item easy experiment submission to remote nodes
  \item experiment state and progress information easily accessible
  \item access to computing node resources information
  \item customizable notification and autotermination system
  \item additional conveniences: drag drop, experiment template, customizable
    parameter value plotting 
  \end{itemize}
\end{frame}

\section{Demo}

\begin{frame}{Demo}
  Live demonstration of the final product including:
  %We have created video demo and would like to show it now.

  \begin{enumerate}
  \item Cluster \& experiment configuration
  \item Experiment submission
  \item Fetching updates/results
  \item Viewing plots
  \end{enumerate}

  Focus on the GUI this time.
\end{frame}
\begin{frame}{Recap}
  In essence the product is a Python-based tool that enables computational
  researchers to conduct their research more effectively.
  \begin{itemize}
  \item It utilizes SSH and TCP sockets to distribute computational
    experiments into remote computer nodes.
  \item It is framework agnostic in that it permits the use of a very wide
    variety of tools to actually conduct the computing needed (Theano, Torch,
    Scipy, Matlab, R, C++, whatever).
  \end{itemize}
\end{frame}
\begin{frame}{User feedback}
We utilized lots of end user testing and received great feedback from
Simo Tuomisto, a Triton admin:

\begin{itemize}
  \item Thought that our design and architecture was well devised.
  \item Offered performance improvement ideas.
  \item In general he liked our usability.
\end{itemize}
\end{frame}
\begin{frame}{User feedback}
From our peer team we received feedback as well.

Thus we did several
\begin{itemize}
  \item bug fixes
  \item usability improvements
  \item improvements to project documentation and the user guide
\end{itemize}
\end{frame}
\begin{frame}{User feedback}
Pyry also gave us great feedback during development,
regarding requirements and usability etc.

In the end he claimed that
\begin{itemize}
  \item the end product was surprisingly professional
  \item the usability was excellent
  \item and that our team had expertise
\end{itemize}
\end{frame}

\section{Quality} \bgset{gfx/neural1__bgmod.jpg}

\begin{frame}{Quality}
  Quality assurance was based on

  \begin{itemize}
  \item the DoD, and
  \item our QA practices
  \end{itemize}

  We updated our DoD and practices several times.
\end{frame}
\begin{frame}{Definition of done}
  We defined \Done{} at the BI, sprint and project level.
    
  \begin{itemize}
  \item BI level: unit tests, code confromity, commenting, documentation, peer review
  \item Sprint level: BIs \Done{}, unit and system tests ok, sprint goal achieved.
  \item Project level: all sprints \Done{} and PO satisfied.
  \end{itemize}
\end{frame}
\begin{frame}{QA practices}
  Used QA practices and tools:
  \begin{itemize}
  \item Unit test framework
  \item Commenting \& documentation
  \item System testing
  \item Peer review
  \end{itemize}
\end{frame}
\begin{frame}{Quality attributes}
  Added emphasis on these attributes:

  \begin{itemize}
  \item Usability
  \item Reliability
  \item Extendability
  \item Performance
  \end{itemize}
\end{frame}

\section{Process}

\begin{frame}{Process overview}
  \begin{itemize}
  \item Each developer as a sprint team leader in turn
  \item Teamwork sessions almost every Wed \& Fri at Maari
  \item Scrum as required
  \item Tried to employ professional SD practices
  \item Tools: Github, Agilefant, Flowdock, WhatsApp, Google Calendar,
    ShareLaTeX, Skype, Floobits
  \end{itemize}
\end{frame}
\begin{frame}{Schedule}
  \begin{table} \small
  \begin{tabular}{lll}
    Time          & Event                     & Participants     \\ \hline
    30.10. 16-18  & Project kickoff           & team, PO         \\
    13.11. 15-17  & S0 demo                   & team, Coach      \\
    16.11. 11-13  & S1 planning               & team, PO         \\
    04.12. 16-17  & S1 \& progress review     & team, PO, Coach  \\
    04.12. 17-18  & S2 planning               & team, PO         \\
    13.01. 19-20  & S2 review \& S3 planning  & team, PO         \\
    01.02. 14-16  & S3 review \& S4 planning  & team, PO         \\
    29.02. 13-14  & S4 \& progress review     & team, PO, Coach  \\
    29.02. 14-15  & S5 planning               & team, PO         \\
    30.03. 16-18  & S5 review \& S6 planning  & team, PO         \\
    13.04. 16-17  & S6 \& project review      & team, PO, Coach  \\
    19.04. 16-20  & Quality award \& party    & team, Coach
  \end{tabular}
  \end{table}
\end{frame}
\begin{frame}{Sprints}
  \begin{itemize}
  \item S1: Develop a prototype that offers the most basic functionality via a CLI \Done{}
  \item S2: Develop a stable version for end user testing \Done{}
  \item S3: Finish asynchronous system functionality and create a GUI mockup \Done{}
  \item S4: Publish Neronet as an open source project \Done{}
  \item S5: Finish Neronet 1.0 \Done{}
  \item S6: Wrap things up \Done{} % Grab the quality award! ?
  \end{itemize}

  There were few stories that were deferred or completely cancelled, but no
  unpleasant surprises.
\end{frame}

\section{Effort}

\begin{frame}{Effort}
Spent and budgeted effort in hours by team member and sprint:
\begin{table} \small
  \begin{tabular}{l|rrrrrrrrrrr}
    S &  Samuel &   Teemu &   Tuomo &   Joona &    Iiro &  Matias  \\ \hline
    0 &  140/50 &   36/35 &   45/35 &   40/35 &   36/35 &   43/35  \\
    1 &   52/30 &   37/33 &   42/33 &   46/33 &   32/33 &   37/33  \\
    2 &   42/30 &   27/33 &   41/33 &   25/33 &   27/33 &   30/33  \\
    3 &   24/15 &   28/33 &   14/33 &   23/33 &   20/33 &   27/33  \\
    4 &   37/15 &   35/33 &   28/33 &   32/33 &   25/33 &   32/33  \\
    5 &   18/20 &   31/40 &   49/40 &   16/40 &   39/40 &   29/40  \\
    6 &   17/15 &    9/18 &   10/18 &   16/18 &   16/18 &   10/18  \\ \hline
      & 330/175 & 203/225 & 229/225 & 198/225 & 195/225 & 208/225  \\
  \end{tabular}
  \end{table}
\end{frame}
\begin{frame}{Burnup}
  \begin{figure}
    \centering\includegraphics[width=1\columnwidth]{gfx/burnup.jpg}
  \end{figure}
\end{frame}

\section{Remarks} \bgset{gfx/neural3__bgmod.jpg}

\begin{frame}{Future}
  The goal was to develop a tool with a bright future.

  \begin{itemize}
  \item Extensive documentation, manual and start guide
  \item Published as an open source project
  \item Located in Github and installable via \texttt{pip}
  \end{itemize}
\end{frame}
\begin{frame}{Complicating factors}
  Complicating factors
  \begin{itemize}
  \item PO mostly in London
  \item Little additional CSE expertise available (only the Coach)
  \item Very difficult domain
  \item Inexperienced team
  \item Only five developers
  \item The PO nor the scrum master had participated in the course before
  \end{itemize}
\end{frame}
\begin{frame}{Simplifying factors}
  Simplifying factors
  \begin{itemize}
  \item Our PO was straightforward, very motivated and gave great feedback
  \item Our coach also gave us valuable support and feedback
  \item We had high motivation
  \end{itemize}
\end{frame}
\begin{frame}{Additional points}
  Additional points
  \begin{itemize}
  \item Sprint team leader system
  \item User guide/manuals enhanced communication
  \item Early end user testing
  \item Open source future focus
  \end{itemize}
\end{frame}
\begin{frame}{Summary}
  Summary:
  \begin{itemize}
  \item The process was smooth
  \item All sprints were successful
  \item More than fulfilled the initial vision
  \item Surpassed PO's expectations
  \item We learned and had fun together :)
  \end{itemize}
\end{frame}
\end{document}
