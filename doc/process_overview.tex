%% Define document class, variables and frontmatter
\documentclass{beamer}
\title{Process Overview}
\def\showoutlineatsection{1}
%%% Variables, functions and other settings
%% Define beamer theme
\usetheme{Goettingen}
%% Packages
%\usepackage[utf8]{inputenc}
%\usepackage[T1]{fontenc}
%% Define basic variables
\subtitle{}
\author{Neronet}
\institute[]{
  \emph{
    Toolbox for managing the training \\
    neural networks
  } \\[0.5cm]
  CSE-C2610 \\
  Software Project \\[0.2cm]
  Aalto University
}
\date{\today{}}
\subject{Software engineering}
%% Colors and visuals
% Specify link and URL colors
\hypersetup{colorlinks,linkcolor=,urlcolor=magenta}
% Replace navigation symbols with a logo
\logo{\includegraphics[height=1cm]{gfx/logo_aalto.png}}
\setbeamertemplate{navigation symbols}{\insertlogo}
%% Functions
% Define short set and clear background commands
\newcommand{\bgset}[1]{\usebackgroundtemplate{
  \includegraphics[width=\paperwidth,height=\paperheight]{#1}}}
\newcommand{\bgclear}{\usebackgroundtemplate{}}
\ifdefined \showoutlineatsection
  % Have LaTeX render an outline frame just before each new section
  %\setbeamercolor{section in toc}{fg=alerted text.fg}
  %\setbeamercolor{section in toc shaded}{fg=structure}
  \setbeamertemplate{section in toc shaded}[default][60]
  \AtBeginSection[]{
    %\bgset{gfx/neural3__bgmod.jpg}
    \begin{frame}<beamer>{Outline}
      \tableofcontents[currentsection]
    \end{frame}
    \bgset{gfx/neural1__bgmod.jpg}}
\fi
%%% Main document content
\begin{document}
%% Title page
\bgset{gfx/neural2__bgmod.jpg}
\begin{frame}
  \titlepage
\end{frame}
%% Outline page
\bgset{gfx/neural3__bgmod.jpg}
\begin{frame}{Outline}
  \tableofcontents
\end{frame}

%% Content pages
\section{Schedule}
\subsection{Plans}
\begin{frame}{Sprints}
  \begin{table} \small
  \begin{tabular}{l|lllllllllll}
    S & Start   & End    & D  & Sa & Te & Tu & Jo & Ju & Ii & Ma \\ \hline
    0 & 19.10.  & 13.11. & 25 & 50 & 35 & 35 & 35 & 35 & 35 & 35 \\
    1 & 13.11.  & 4.12.  & 21 & 30 & 33 & 33 & 33 & 33 & 33 & 33 \\
    2 & 4.12.   & 13.1.  & 38 & 30 & 33 & 33 & 33 & 33 & 33 & 33 \\
    3 & 13.1.   & 1.2.   & 21 & 15 & 33 & 33 & 33 & 33 & 33 & 33 \\
    4 & 1.2.    & 29.2.  & 28 & 15 & 33 & 33 & 33 & 33 & 33 & 33 \\
    5 & 29.2.   & 21.3.  & 21 & 15 & 33 & 33 & 33 & 33 & 33 & 33 \\
    6 & 21.3.   & 11.4.  & 21 & 20 & 25 & 25 & 25 & 25 & 25 & 25 \\
  \end{tabular}
  \end{table}

  Note

  \begin{itemize}
  \item S2 includes exams (7.-18.12.) and holidays (23.12.-1.1.)
  \item S4 includes exams (15.-19.2.)
  \item S6 includes exams (4.-9.4) and is reserved for mainly polishing \& documenting for final review (11.-13.4.)
  \end{itemize}
\end{frame}
\begin{frame}{Events}
  \begin{table} \small
  \begin{tabular}{lll}
    Time          & Event              & Participants       \\ \hline
    30.10. 16-18  & Project kickoff    & team + PO          \\
    13.11. 15-17  & Sprint 0 demo      & team + Coach       \\
    16.11. 11-13  & Sprint 1 planning  & team + PO          \\
    04.12. 16-17  & Progress review 1  & team + PO + Coach  \\
    04.12. 17-18  & Sprint change 1    & team + PO          \\
    13.01. 19-20  & Sprint change 2    & team + PO          \\
    01.02. 14-16  & Sprint change 3    & team + PO          \\
  \end{tabular}
  \end{table}

  All events, locations, agendas and other details are uptodate in
  \href{https://calendar.google.com/calendar/embed?src=0a7g91rk53ar9v75o089icgi8s\%40group.calendar.google.com\&ctz=Europe/Helsinki}{Google Calendar}.
\end{frame}
% \subsection{Realized}
% \begin{frame}{Workload}
%   Diagram or table of realized workload over sprints
% \end{frame}
% \begin{frame}{Week 1}
%   Summary diagram or table of week's activity from Agilefant data
% \end{frame}
% \begin{frame}{Week 2}
%   Summary diagram or table of week's activity from Agilefant data
% \end{frame}
% \begin{frame}{Week 3}
%   Summary diagram or table of week's activity from Agilefant data
% \end{frame}
% add other weeks!
\section{Recurring events} %Concrete, short instructions (e.g. when, organizer \& participants, how, results) for at least the following activities.
\subsection{Overview}
\begin{frame}{Overview}
  Recurring events:

  \begin{itemize}
  \item Sprint planning
  \item Sprint review
  \item Sprint retrospective
  \item \emph{Daily} scrums
  \item Teamwork sessions
  \end{itemize}
\end{frame}
\subsection{Events}
\begin{frame}{Sprint planning}
  A sprint planning session is organized at the start of each sprint.
  \begin{enumerate}
  \item Before the session
    \begin{itemize}
    \item the PO makes sure the \alert{product backlog} contains an ordered list
    of items with a description and a number depicting business value
    \item the team plays planning poker to define effort estimates
    (\alert{story points}) for each BI
    \end{itemize}
  \item During it the team and the PO
    \begin{itemize}
    \item briefly define the increment's purpose, the \alert{sprint goal}
    \item move BIs from the product backlog to the \alert{sprint backlog}
    \end{itemize}
  \item After it, the team
    \begin{itemize}
    \item chews the BIs into \alert{smaller tasks}
    \item assigns effort estimates on the tasks by \alert{planning poker}
    \item assigns a \alert{developer and a reviewer} to each task
    \end{itemize}
  \end{enumerate}
\end{frame}
\begin{frame}{Sprint review}
  At the end of each sprint, we
  \begin{itemize}
  \item demonstrate the stories we were able to get \emph{done}
  \item adapt the product backlog based on the results, if needed
  \end{itemize}
\end{frame}
\begin{frame}{Sprint retrospective}
  After the sprint review, we
  \begin{itemize}
  \item evaluate and rank teamwork practices
  \item discuss how teamwork could be improved
  \item remove/replace any bad practices
  \item plan implementation of new improvements
  \item give feedback to sprint team leader
  \end{itemize}
\end{frame}
\begin{frame}{Daily scrums}
  On Wednesdays and Fridays we have a scrum in which everyone quickly
  explains what
  \begin{itemize}
  \item they did since last Scrum
  \item problems they have encountered
  \item they plan to do before the next Scrum
  \end{itemize}

  Work plans are adjusted depending on input.
\end{frame}
\begin{frame}{Teamwork sessions}
  Most weeks, we'll

  \begin{itemize}
  \item have a Scrum and a 6h session on Wednesdays
  \item have a Scrum and a 5h session on Fridays
  \item do some individual work remotely to cover up any missed
    sessions
  \end{itemize}

  Team sessions are mainly held in Maari. The team leader leads the sessions.
\end{frame}
\section{Practices} % Concrete, short instructions (e.g. when, by whom, how, results) for other main practices and tools.
\subsection{Overview}
\begin{frame}{Overview}
  Used practices and tools:

  \begin{itemize}
  \item Testing \& quality assurance: DoD
  \item Communication: Email, Flowdock, Hangout/Skype, WhatsApp
  \item Backlog management: Agilefant
  \item Time tracking: Agilefant
  \item Version control: GitHub
  \item Collaboration: Floobits, ShareLaTeX, Google Drive
  \item Motivation: Team Spirit Recap
  %\item Any other practices?
  \end{itemize}
\end{frame}
\subsection{Information}
\begin{frame}{Quality assurance}
  We guarantee quality by making

  \begin{itemize}
  \item sure team members adhere to the \href{https://github.com/smarisa/sdpt11/raw/master/doc/definition_of_done.pdf}{DoD}.
  \item each member responsible for the quality of the code he reviewed.
  \item the PO is responsible for the business value of sprint goals and
  BIs and for making sure the team understands them.
  \end{itemize}
\end{frame}
\begin{frame}{Communication}
  We use the following channels:

  \begin{itemize}
  \item Email - communication that involves the PO and Coach
  \item Flowdock - general forum for everyday discussion
  \item WhatsApp/Phone - urgent team communication
  \item Skype/Hangout - remote teamworking sessions
  \end{itemize}

  The sprint team leader communicates with the PO and Coach.
\end{frame}
\begin{frame}{Backlog management}
  Agilefant is used for all backlogs.

  \begin{itemize}
  \item Version 1 - the \href{https://cloud.agilefant.com/smarisa/editProject.action?projectId=154985}{product backlog} 
  \item Sprint 0-6 - the sprint backlogs
  \end{itemize}
\end{frame}
\begin{frame}{Time tracking}
  We track our worktime with Agilefant. We log each work session
  duration to the story or task we worked on.
\end{frame}
\begin{frame}{Version control}
  We use Git with GitHub and branches:

  \begin{itemize}
  \item \href{https://github.com/smarisa/sdpt11}{stable} - tested and working version
  \item \href{https://github.com/smarisa/sdpt11}{sprint} - increment work in progress
  \item story\emph{X} - story work in progress
  \end{itemize}

  Our development process has four steps:

  \begin{enumerate}
  \item We assign a developer and a reviewer for each story
  \item The developer solves the story in a new branch
  \item Then he asks the reviewer for a merge review
  \item The reviewer determines whether the work meets the story
  requirements and the DoD
    \begin{itemize}
    \item if not, he asks the developer to continue working on it
    \item if yes, he merges the story branch to the sprint branch and
    the developer marks the story as \emph{done}
    \end{itemize}
  \end{enumerate}
\end{frame}
\begin{frame}{Collaboration}
  When we work simultaneously on the same documents we use Floobits,
  ShareLaTeX, or Google Drive depending on the document.

  Floobits is connected to a Git repo clone which facilitates when working
  with many files. It is particularly suitable for collaborative code level
  planning and code reviews.
\end{frame}
\begin{frame}{Motivation}
  We have three main practices to maintain motivation:
  \begin{itemize}
  \item Regular review of the \emph{six tactics}
  \item Regular review of our \emph{team spirit recap}
  \item Regular discussion on problems (retrospectives)
  \end{itemize}
\end{frame}
\begin{frame}{Motivation}{Six tactics}
  \begin{enumerate}
  \item Create common goals
  \item Focus on facts
  \item Develop multiple alternatives
  \item Maintain a balanced power structure
  \item Seek consensus with qualification
  \item Use humour
  \end{enumerate}

  \tiny Eisenhardt K M, Kahwajy J L, and Bourgeois III L J (1997)
  How Management Teams Can Have a Good Fight,
  Harvard Business Review, Vol. 4, pp. 77-85.
\end{frame}
\begin{frame}{Motivation}{Team spirit recap}
  Mission: Why we exist
  \begin{itemize}
  \item Create useful software for Pyry (and others)
  \item We are doing this project to learn {(\tiny software development,
    requirements engineering, architecture, project management, quality
    assurance, Scrum, communication with client)}
  \item We want grade five and the quality award
  \end{itemize}
\end{frame}
\begin{frame}{Motivation}{Team spirit recap}
  Values: What we believe in and how we will behave
  \begin{itemize}
  \item Superior quality
  \item Self-development
  \item Respect
  \item Achievement
  \end{itemize}
\end{frame}
\begin{frame}{Motivation}{Team spirit recap}
  Vision: What we want to be
  \begin{itemize}
  \item We want to see ourselves as the best of the course teams
  \item We want to win the Quality award!
  \item We want to get grade 5+.
  \item We want to get an awesome reference (GitHub repo) that we can market
    on our future job applications.
  \item We want our tool to serve people in such a way that a community of
    users develops around it and continues it's development. We want to
    launch a successful opensource project, which we can speak proudly of
    even years from now.
  \end{itemize}
\end{frame}
\begin{frame}{Motivation}{Team spirit recap}
  \begin{itemize}
  \item Objective: Ace the course and develop a very useful and popular tool
  \item Scope: See product vision
  \item Advantage: We have high motivation, we meet in person every week, we
    have an active and responsible Scrum Master
  \end{itemize}
\end{frame}
% Any other practices?
\end{document}