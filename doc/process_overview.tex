%%% Variables, functions and other settings
%% Define document class and theme
\documentclass{beamer}
\usetheme{Goettingen}
%% Define basic variables
\title{Process Overview}
\subtitle{}
\author{Team 11}
\institute[]{
  Project 2 \\
  Toolbox for managing the training \\
  neural networks (Pyry Takala) \\[0.3cm]
  CSE-C2610 Software Project \\
  Aalto University
}
\date{14th Oct 2015}
\subject{Software engineering}
%% Logo
% Replace navigation symbols with a logo
\logo{\includegraphics[height=1cm]{gfx/logo_aalto.png}}
\setbeamertemplate{navigation symbols}{\insertlogo}
%% Functions
% Define short set and clear background commands
\newcommand{\bgset}[1]{\usebackgroundtemplate{
  \includegraphics[width=\paperwidth,height=\paperheight]{#1}}}
\newcommand{\bgclear}{\usebackgroundtemplate{}}
% Have LaTeX render an outline frame just before each new section
\AtBeginSection[]{
  \bgset{gfx/neural3__bgmod.jpg}
  \begin{frame}<beamer>{Outline}
    \tableofcontents[currentsection]
  \end{frame}
  \bgset{gfx/neural1__bgmod.jpg}}
%%% Main document content
\begin{document}
%% Title page
\bgset{gfx/neural2__bgmod.jpg}
\begin{frame}
  \titlepage
\end{frame}
%% Outline page
\bgset{gfx/neural3__bgmod.jpg}
\begin{frame}{Outline}
  \tableofcontents
\end{frame}
%% Content pages
\section{Schedule}
\subsection{Sprints}
\begin{frame}{Sprints}
  \begin{table} \small
  \begin{tabular}{l|lllllllllll}
    S & Start   & End    & D  & Sa & Te & Tu & Jo & Ju & Ii & Ma \\ \hline
    0 & 19.10.  & 13.11. & 25 & 50 & 35 & 35 & 35 & 35 & 35 & 35 \\
    1 & 13.11.  & 4.12.  & 21 & 30 & 33 & 33 & 33 & 33 & 33 & 33 \\
    2 & 4.12.   & 11.1.  & 38 & 30 & 33 & 33 & 33 & 33 & 33 & 33 \\
    3 & 11.1.   & 1.2.   & 21 & 15 & 33 & 33 & 33 & 33 & 33 & 33 \\
    4 & 1.2.    & 29.2.  & 28 & 15 & 33 & 33 & 33 & 33 & 33 & 33 \\
    5 & 29.2.   & 21.3.  & 21 & 15 & 33 & 33 & 33 & 33 & 33 & 33 \\
    6 & 21.3.   & 11.4.  & 21 & 20 & 25 & 25 & 25 & 25 & 25 & 25 \\
  \end{tabular}
  \end{table}

  Note

  \begin{itemize}
  \item S2 includes exams (7.-18.12.) and holidays (23.12.-1.1.)
  \item S4 includes exams (15.-19.2.)
  \item S6 includes exams (4.-9.4) and is reserved for mainly polishing \& documenting for final review (11.-13.4.)
  \end{itemize}
\end{frame}
\begin{frame}{Workload}
  Diagram or table of realized workload over sprints
\end{frame}
\subsection{Weeks}
\begin{frame}{Week 1}
  Summary diagram or table of week's activity from Agilefant data
\end{frame}
\begin{frame}{Week 2}
  Summary diagram or table of week's activity from Agilefant data
\end{frame}
\begin{frame}{Week 3}
  Summary diagram or table of week's activity from Agilefant data
\end{frame}
% add other weeks!
\section{Recurring events} %Concrete, short instructions (e.g. when, organizer \& participants, how, results) for at least the following activities.
\subsection{Overview}
\begin{frame}{Overview}
  Recurring events:

  \begin{itemize}
  \item Sprint planning
  \item Sprint review
  \item Sprint retrospective
  \item "Daily" scrums
  \item Teamwork sessions
  \end{itemize}
\end{frame}
\subsection{Events}
\begin{frame}{Sprint planning}
  A sprint planning session is organized at the start of each sprint.
  \begin{enumerate}
  \item Before the session
    \begin{itemize}
    \item the PO makes sure the product backlog contains an ordered list
    of items with a description and a number depicting business value
    \item the team plays planning poker to define effort estimates
    (story points) for each BI
    \end{itemize}
  \item During it the team and the PO
    \begin{itemize}
    \item briefly define the increment's purpose, the \alert{sprint goal}
    \item move BIs from the product backlog to the \alert{sprint backlog}
    \end{itemize}
  \item After it, the team
    \begin{itemize}
    \item chews the BIs into smaller bits
    \item assigns effort estimates on the bits (planning poker)
    \item assigns a developer and a reviewer to each bit
    \end{itemize}
  \end{enumerate}
\end{frame}
\begin{frame}{Sprint review}
  At the end of each sprint, we
  \begin{itemize}
  \item demonstrate the stories we were able to get \emph{done}
  \item adapt the product backlog based on the results, if needed
  \end{itemize}
\end{frame}
\begin{frame}{Sprint retrospective}
  After the sprint review, we
  \begin{itemize}
  \item evaluate and rank teamwork practices
  \item discuss how teamwork could be improved
  \item remove/replace any bad practices
  \item plan implementation of new improvements
  \end{itemize}
\end{frame}
\begin{frame}{Daily scrums}
  On Mondays and Fridays we have a scrum in which everyone quickly
  explains what
  \begin{itemize}
  \item they did since last Scrum
  \item problems they have encountered
  \item they plan to do next
  \end{itemize}
\end{frame}
\begin{frame}{Teamwork sessions}
  Most weeks, we'll have

  \begin{itemize}
  \item a quick remote Scrum on Mondays
  \item a 2h session on Wednesdays
  \item a Scrum and a 7h session on Fridays (Tuomo would prefer splitting this)
  \end{itemize}

  In addition, we do individual work remotely.
\end{frame}
\section{Practices} % Concrete, short instructions (e.g. when, by whom, how, results) for other main practices and tools.
\subsection{Overview}
\begin{frame}{Overview}
  Used practices and tools:

  \begin{itemize}
  \item Testing \& quality assurance: DoD
  \item Communication: Flowdock, Email, WhatsApp
  \item Backlog management: Agilefant
  \item Time tracking: Agilefant
  \item Version control: GitHub
  %\item Any other practices?
  \end{itemize}
\end{frame}
\subsection{Information}
\begin{frame}{Quality assurance}
  We guarantee quality by making

  \begin{itemize}
  \item sure team members adhere to the DoD.
  \item each member responsible for the quality of the code he reviewed.
  \item the PO is responsible for the business value of sprint goals and
  BIs and for making sure the team understands them.
  \end{itemize}

  The DoD is available here: FIXME
\end{frame}
\begin{frame}{Communication}
  We use the following channels:

  \begin{itemize}
  \item Flowdock - general forum for everyday discussion
  \item Email - communication with PO and Coach
  \item WhatsApp/Phone - urgent team communication
  \end{itemize}

  The SM communicates with the PO and Coach.
\end{frame}
\begin{frame}{Backlog management}
  Agilefant is used for all backlogs.
  
  When defining BIs we specify

  \begin{enumerate}
  \item story points (by team)
  \item value (by PO)
  \item effort estimates (by team)
  \item initial assignees (by team)
  \end{enumerate}
\end{frame}
\begin{frame}{Time tracking}
  We track our worktime with Agilefant by logging each session
  duration as effort spent to a story or task.
\end{frame}
\begin{frame}{Version control}
  We use Git with GitHub with branches:

  \begin{itemize}
  \item stable - tested and working version
  \item sprint - increment work in progress
  \item storyX - story work in progress
  \end{itemize}

  Our development process has four steps:

  \begin{enumerate}
  \item We assign a developer and reviewer for each story
  \item The story assignee develops the story in a new branch
  \item He makes a merge request once his work is ready for review
  \item The reviewer merges the story to the sprint only when it meets the DoD
  \end{enumerate}
\end{frame}
% Any other practices?
\end{document}
