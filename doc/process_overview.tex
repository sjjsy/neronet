%%% Variables, functions and other settings
%% Define document class and theme
\documentclass{beamer}
\usetheme{Goettingen}
%% Define basic variables
\title{Process Overview}
\subtitle{}
\author{Team 11}
\institute[]{
  Project 2 \\
  Toolbox for managing the training \\
  neural networks (Pyry Takala) \\[0.3cm]
  CSE-C2610 Software Project \\
  Aalto University
}
\date{14th Oct 2015}
\subject{Software engineering}
%% Logo
% Replace navigation symbols with a logo
\logo{\includegraphics[height=1cm]{gfx/logo_aalto.png}}
\setbeamertemplate{navigation symbols}{\insertlogo}
%% Functions
% Define short set and clear background commands
\newcommand{\bgset}[1]{\usebackgroundtemplate{
  \includegraphics[width=\paperwidth,height=\paperheight]{#1}}}
\newcommand{\bgclear}{\usebackgroundtemplate{}}
% Have LaTeX render an outline frame just before each new section
\AtBeginSection[]{
  \bgset{gfx/neural3__bgmod.jpg}
  \begin{frame}<beamer>{Outline}
    \tableofcontents[currentsection]
  \end{frame}
  \bgset{gfx/neural1__bgmod.jpg}}
%%% Main document content
\begin{document}
%% Title page
\bgset{gfx/neural2__bgmod.jpg}
\begin{frame}
  \titlepage
\end{frame}
%% Outline page
\bgset{gfx/neural3__bgmod.jpg}
\begin{frame}{Outline}
  \tableofcontents
\end{frame}
%% Content pages
\section{Schedule}
\subsection{Sprints}
\begin{frame}{Sprints}
  Diagram or table of sprints over time
;Sprint;Start;End;Days;Samuel;Matias;Iiro;Joona;Teemu;Tuomo;Juho;Comments
;Sprint 0;19.10.;13.11.;25;50;35;35;35;35;35;35;
;Sprint 1;13.11.;4.12.;21;35;34;34;34;34;34;34;Sprint change done before project review (7.-9.12.)
;Sprint 2;4.12.;11.1.;38;20;34;34;34;34;34;34;Exams (7.-18.12.) and holidays (23.12.-1.1.)
;Sprint 3;11.1.;1.2.;21;15;34;34;34;34;34;34;
;Sprint 4;1.2.;29.2.;28;15;34;34;34;34;34;34;Exams (15.-19.2.)
;Sprint 5;29.2.;21.3.;21;20;34;34;34;34;34;34;
;Sprint 6;21.3.;11.4.;21;20;20;20;20;20;20;20;Only polishing & documenting for final review (11.-13.4.) amid exams (4.-9.4)
\end{frame}
\begin{frame}{Workload}
  Diagram or table of workload over sprints
\end{frame}
\subsection{Weeks}
\begin{frame}{Week 1}
  Diagram or table of week's activity
\end{frame}
\begin{frame}{Week 2}
  Diagram or table of week's activity
\end{frame}
\begin{frame}{Week 3}
  Diagram or table of week's activity
\end{frame}
% add other weeks!
\section{Recurring events} %Concrete, short instructions (e.g. when, organizer \& participants, how, results) for at least the following activities.
\subsection{Overview}
\begin{frame}{Overview}
  Recurring events:

  \begin{itemize}
  \item Sprint Planning
  \item Sprint Review
  \item Sprint Retrospective
  \item Daily Scrums
  \item Teamwork sessions
  \end{itemize}
\end{frame}
\subsection{Events}
\begin{frame}{Sprint Planning}
  At the beginning of each sprint

  \begin{itemize}
  \item We define with the PO:
    \begin{itemize}
    \item the Sprint Goal (broad definition of the next increment)
    \item the Sprint Backlog (an ordered list of BLIs included in the sprint)
    \end{itemize}
  \item The team starts planning how the chosen work will get done.
  \end{itemize}
\end{frame}
\begin{frame}{Sprint Review}
  At the end of each sprint

  \begin{itemize}
  \item We
    \begin{itemize}
    \item demonstrate the stories we were able to get done
    \item adapt the product backlog based on the results, if needed
    \end{itemize}
  \end{itemize}
\end{frame}
\begin{frame}{Sprint Retrospective}
  \begin{itemize}
  \item When: after the sprint review
  \item What: We
    \begin{itemize}
    \item discuss how teamwork could be improved
    \item evaluate and rank practices, replace bad ones, if needed
    \item plan implementation of improvements
    \end{itemize}
  \end{itemize}
\end{frame}
\begin{frame}{Daily Scrums}
  \begin{itemize}
  \item When: On Mondays and Fridays
  \item What: Everyone quickly explains what
    \begin{itemize}
    \item they did since last Scrum
    \item they plan to do next
    \item problems have arisen
    \end{itemize}
  \end{itemize}
\end{frame}
\begin{frame}{Teamwork sessions}
  Most weeks, we'll have

  \begin{itemize}
  \item a quick remote Scrum on Mondays
  \item a 2h session on Wednesdays
  \item a Scrum and a 7h session on Fridays
  \end{itemize}

  In addition, we work remotely individually.
\end{frame}
\section{Practices} % Concrete, short instructions (e.g. when, by whom, how, results) for other main practices and tools.
\subsection{Overview}
\begin{frame}{Overview}
  Used practices and tools:

  \begin{itemize}
  \item Testing \& quality assurance: DoD
  \item Communication: Flowdock, Email, WhatsApp
  \item Backlog management: Agilefant
  \item Time tracking: Agilefant
  \item Version control: GitHub
  \item Any other practices?
  \end{itemize}
\end{frame}
\subsection{Information}
\begin{frame}{Quality assurance}
  We guarantee quality by making

  \begin{itemize}
  \item sure team members adhere to the DoD.
  \item each member responsible for the quality of the code he reviewed.
  \item the PO is responsible for the quality of goals and value of stories.
  \end{itemize}

  The DoD is available here: FIXME
\end{frame}
\begin{frame}{Communication}
  We use the following channels:

  \begin{itemize}
  \item Flowdock – general forum for everyday discussion
  \item Email – communication with PO and Coach
  \item WhatsApp/Phone – urgent team communication
  \end{itemize}

  The SM communicates with the PO and Coach.
\end{frame}
\begin{frame}{Backlog management}
  Agilefant is used for all backlogs.
  
  When defining BIs we specify

  \begin{enumerate}
  \item story points (by team)
  \item value (by PO)
  \item effort estimates (by team)
  \item initial assignees (by team)
  \end{enumerate}
\end{frame}
\begin{frame}{Time tracking}
  We track our worktime with Agilefant by logging each session
  duration as effort spent to a story or task.
\end{frame}
\begin{frame}{Version control}
  We use Git with GitHub with branches:

  \begin{itemize}
  \item stable – tested and working version
  \item sprint – increment work in progress
  \item storyX – story work in progress
  \end{itemize}

  Our development process has four steps:

  \begin{enumerate}
  \item We assign a developer and reviewer for each story
  \item The story assignee develops the story in a new branch
  \item He makes a merge request once his work is ready for review
  \item The reviewer merges the story to the sprint only when it meets the DoD
  \end{enumerate}
\end{frame}
% Any other practices?
\end{document}
