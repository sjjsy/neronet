%% Define document class, variables and frontmatter
\documentclass{beamer}
\title{Product Vision}
%%% Variables, functions and other settings
%% Define beamer theme
\usetheme{Goettingen}
%% Packages
%\usepackage[utf8]{inputenc}
%\usepackage[T1]{fontenc}
%% Define basic variables
\subtitle{}
\author{Neronet}
\institute[]{
  \emph{
    Toolbox for managing the training \\
    neural networks
  } \\[0.5cm]
  CSE-C2610 \\
  Software Project \\[0.2cm]
  Aalto University
}
\date{\today{}}
\subject{Software engineering}
%% Colors and visuals
% Specify link and URL colors
\hypersetup{colorlinks,linkcolor=,urlcolor=magenta}
% Replace navigation symbols with a logo
\logo{\includegraphics[height=1cm]{gfx/logo_aalto.png}}
\setbeamertemplate{navigation symbols}{\insertlogo}
%% Functions
% Define short set and clear background commands
\newcommand{\bgset}[1]{\usebackgroundtemplate{
  \includegraphics[width=\paperwidth,height=\paperheight]{#1}}}
\newcommand{\bgclear}{\usebackgroundtemplate{}}
\ifdefined \showoutlineatsection
  % Have LaTeX render an outline frame just before each new section
  %\setbeamercolor{section in toc}{fg=alerted text.fg}
  %\setbeamercolor{section in toc shaded}{fg=structure}
  \setbeamertemplate{section in toc shaded}[default][60]
  \AtBeginSection[]{
    %\bgset{gfx/neural3__bgmod.jpg}
    \begin{frame}<beamer>{Outline}
      \tableofcontents[currentsection]
    \end{frame}
    \bgset{gfx/neural1__bgmod.jpg}}
\fi
%%% Main document content
\begin{document}
%% Title page
\bgset{gfx/neural2__bgmod.jpg}
\begin{frame}
  \titlepage
\end{frame}
%% Outline page
\bgset{gfx/neural3__bgmod.jpg}
\begin{frame}{Outline}
  \tableofcontents
\end{frame}

%% Content pages
\section{Why}
% \begin{frame}{Why -- business view}
%   Neural networks have become quite the hot topic when it comes to the machine learning research today. However, the tools available for most researchers for specifying, managing and queuing the neural network experiments are still stone-age. In particular, a good open source solution is still missing. 
  
%   To satisfy the needs of deep learning researchers we have decided to develop an open source toolbox containing the most common necessities for managing the training of neural networks.
% \end{frame}
\begin{frame}{Why -- business view}
  Currently available state-of-the-art tools for research involving heavy
  computation aren't satisfactory.

  Researchers are burdened by practical difficulties like
  \begin{itemize}
  \pause \item managing a queue and history of different experiments
  \pause \item specifying several variations of experiments and running them
  \pause \item getting information about the computing environment
  \pause \item monitoring and controlling progress of ongoing experiments
  \pause \item analysing and comparing the results of experiment variations
  \end{itemize}

  \pause Poor tools lead to ineffective use of man and machine hours.
\end{frame}
\section{What}
%\begin{frame}{What -- product goals}
%  Functionalities
%  \begin{itemize}
%  \item Enabling users to use any of the most common frameworks for specifying their neural networks
%  \item Enabling specifying of neural network experiments and managing a queue of experiments via an internet-based solution
%  \item Enabling user login and setting of individual preferences
%  \item Enabling users to see an overview of all previous experiments
%  \item Enabling  monitoring ongoing experiments by analysing the training log and parameter values as well as giving understandable visualisations of the networks
%  \end{itemize}
%\end{frame}
\begin{frame}{What -- product goals}
  The product's goal is to enable easy
  \begin{enumerate}
  \item specification of experiments and management of queues
  \item batch submission of experiment jobs to computing clusters
  \pause \item monitoring of ongoing experiments' logs and parameter values
  \item access to experiment information during and after the run
  \item configurable notifications on experiment state and progress
  \item configurable criteria for experiment autotermination
  \pause \item logging of experiment history
  \item preferences configuration
  \end{enumerate}  
\end{frame}
\begin{frame}{What -- product goals}
  The goals should be achieved in a generic way suitable for many different
  computational problem areas and experiment types.

  \pause Potential extra goals:
  \begin{enumerate}
  \item visualisations from experiment data
  \item multi-user support
  \item management of experiment queues
  \end{enumerate}

  \pause Nonfunctional requirements:
  \begin{enumerate}
  \item low computational and memory overhead
  \item good usability
  \item easily maintainable and extensible
  \item open source
  \end{enumerate}
\end{frame}
% to develop a toolbox for deep learning research with the following features:
% - framework agnostic, support for at least Torch and Theano
% - easy specification and management of experiments and experiment queues
% - the work from the experiment queues is distributed to a computer cluster
% - easy access to training logs during and after training
% - automatic monitoring of training logs, configurable notifications of progress and state changes
% - configurable criteria for experiment autotermination
% - low computational and memory overhead
% - good usability
% - easily maintainable and extensible
% - open source
\section{For whom}
\begin{frame}{For whom -- users}
  The envisioned users are all individuals who run long lasting computational
  experiments and appreciate progress feedback.

  \pause The potential user segments include for instance:
  \begin{itemize}
  \item Deep learning researchers
  \item Machine learning researchers
  \item Computational physics researchers
  \item Data science practitioners
  \item Enthusiasts \& hobbyists
  \end{itemize}
\end{frame}
\end{document}